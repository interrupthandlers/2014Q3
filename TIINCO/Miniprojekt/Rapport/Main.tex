%Report template by Christian Fischer Pedersen for IT-ONK
%--------------------------------------------------------

%Page layout
%--------------------------------------------------------
\documentclass[a4paper,10pt]{report}
\usepackage{fancyhdr}
\usepackage{geometry}

%Seperated files
%--------------------------------------------------------
\usepackage{subfiles}

%Text typesetting
%--------------------------------------------------------
\usepackage{bookman}                  
\usepackage[T1]{fontenc}              
\usepackage[utf8]{inputenc}         
\usepackage[english]{babel}        
\usepackage[bf,small,raggedright,compact]{titlesec}
\usepackage[parfill]{parskip}
\usepackage{setspace}
\setcounter{secnumdepth}{3}

\usepackage{listings}
\usepackage{color} %red, green, blue, yellow, cyan, magenta, black, white
\definecolor{mygreen}{RGB}{28,172,0} % color values Red, Green, Blue
\definecolor{mylilas}{RGB}{170,55,241}

%Tables
%----------------------------------------------------------
\usepackage{tabularx}
\usepackage{multirow} 
\usepackage{multicol} 
\usepackage{booktabs}

%Misc
%----------------------------------------------------------
\usepackage{cite}
\usepackage{appendix}
\usepackage{amssymb}
\usepackage{url}
\usepackage{graphicx}

%FixMe pakken viser små kommentarer, hvor der skal rettes
%--------------------------------------------------
%Brug med følgende: \fxnote{det her skal uddybes!} 
%Se liste over alle fixMe's: \listoffixmes
%Erstat 'draft' med 'final' for at fjerne alle kommentarer
%--------------------------------------------------
\usepackage[footnote,draft,english,silent,nomargin]{fixme} 

%Begin document
%----------------------------------------------------------
\begin{document}

\lstset{language=Matlab,%
    %basicstyle=\color{red},
    breaklines=true,%
    morekeywords={matlab2tikz},
    keywordstyle=\color{blue},%
    morekeywords=[2]{1}, keywordstyle=[2]{\color{black}},
    identifierstyle=\color{black},%
    stringstyle=\color{mylilas},
    commentstyle=\color{mygreen},%
    showstringspaces=false,%without this there will be a symbol in the places where there is a space
    numbers=left,%
    numberstyle={\tiny \color{black}},% size of the numbers
    numbersep=9pt, % this defines how far the numbers are from the text
    emph=[1]{for,end,break},emphstyle=[1]\color{red}, %some words to emphasise
    %emph=[2]{word1,word2}, emphstyle=[2]{style},    
}

\begin{titlepage}
\begin{center}
{\LARGE \textbf{Meggit Decoder}}\\
{\large \textbf{TIINCO Mini project}}\\

\vspace{4cm}
\textbf{Handed in March 20, 2014}\\~\\ \fxnote{Indtast dato for aflevering.}
\begin{tabular}{ll}
Ivan Grujic & 10454@iha.dk \\
Lasse Brøsted Pedersen & 10769@iha.dk \\
\end{tabular}
\vfill
\textbf{Electrical and Computer Engineering}\\
\textbf{Aarhus University}\\
\textbf{Finlandsgade 22, 8200 Aarhus N, Denmark}
\end{center}
\end{titlepage}

\setcounter{tocdepth}{1}
\tableofcontents

\listoffixmes

\subfile{Introduction}

\subfile{Implementation}

\subfile{Test}

\subfile{Results}

\fxnote{tilføj eller fjern bibitems}
\begin{thebibliography}{9}
\addcontentsline{toc}{chapter}{Bibliography}

\bibitem{essentials}
J. C. Moreira and P. G. Farrell, \emph{Essentials of Error-Control Coding}, John Wiley \& Sons, 2006,
ISBN-13 978-0-470-029260-6

\bibitem{lec7}	
Q. Zhang, \emph{TIINCO Lecture 7: Cyclic block codes}, 2014


\end{thebibliography}

\end{document}