\documentclass[Main]{subfiles}
\begin{document}

\lstset{language=Matlab,%
    %basicstyle=\color{red},
    breaklines=true,%
    morekeywords={matlab2tikz},
    keywordstyle=\color{blue},%
    morekeywords=[2]{1}, keywordstyle=[2]{\color{black}},
    identifierstyle=\color{black},%
    stringstyle=\color{mylilas},
    commentstyle=\color{mygreen},%
    showstringspaces=false,%without this there will be a symbol in the places where there is a space
   % numbers=left,%
   % numberstyle={\tiny \color{black}},% size of the numbers
    %numbersep=9pt, % this defines how far the numbers are from the text
    emph=[1]{for,end,break},emphstyle=[1]\color{red}, %some words to emphasise
    %emph=[2]{word1,word2}, emphstyle=[2]{style},    
}


\chapter{Encoder}
An encoder has the task of transforming the message vector into the codeword. In this project the following additional requirements apply to the encoder function:

\begin{itemize}
\item The function has two inputs
	\begin{itemize}
	\item Generator polynomial
	\item Message vector
	\end{itemize}
\item The function has one output
	\begin{itemize}
	\item Code vector
	\end{itemize}
\item The code vector must be in systematic form
\end{itemize} 

\section{Cyclic codes in systematic form}
This section explains the theory of constructing cyclic codes in systematic form and is based on \emph{Essentials of Error Control Coding}\cite{essentials} section 3.4

The regular procedure for constructing cyclic codewords is to multiply the message polynomial $m(X)$ with the generator polynomial $g(X)$ to obtain the codeword $c(X)$. However this encoding process produces non-systematic codewords, which is not preferable when implementing a decoder in hardware. Therefore the systematic encoding procedure is introduced.

Given a message polynomial of the form:

{\centering 
$m(X) = m_0 + m_1X + ... + m_{k-1}X^{k-1}$ \par}

the systematic codeword can be obtained by performing the following steps.

\begin{enumerate}
\item The polynomial $X^{n-k}m(X) = m_0X^{n-k} + m_1X^{n-k+1} + ... +m_{k-1}X^{n-1}$ is formed and divided with the generator polynomial $g(X)$:
\begin{equation} \label{eq:XrdividedByGenerator}
X^{n-k}m(X) = q(X)g(X)+p(X)
\end{equation}

\item When reordering equation \ref{eq:XrdividedByGenerator} the following is obtained:
\begin{equation} \label{eq:XrdividedByGeneratorReoreded}
X^{n-k}m(X)+ p(X)=q(X)g(X)
\end{equation}

\item $X^{n-k}m(X)+p(X)$ is a code polynomial since it is a factor of $g(X)$. Furthermore the systematic form is verified by seeing that the term $X^{n-k}m(X)$ is the message vector right-shifted n-k times, and that $p(X)$ is the redundancy polynomial which occupies the lower degree terms of the polynomial expression of the codeword $c(X)$:

{\centering 
$c(X) = p_0 + p_1X + ... + p_{n-k}X^{n-k-1} + m_0X^{n-k} + m_1X^{n-k+1}+...+m_{k-1}X^{n-1}$ \par}

\end{enumerate}

\section{Implementation}
The encoder has been implemented in a Matlab function called  \code{EncodeCyclicSystematic()}, which takes the generator polynomial and message vector as inputs and returns the codeword as output.

The algorithm implemented by the function conforms to the steps described in the previous section. The code segments shown here display the most important parts of the source code. The complete function can be found in Appendix \ref{App:SourceCode}. 

\begin{enumerate}
\item The polynomial $X^{n-k}m(X)$ is formed and divided by $g(X)$:
\begin{lstlisting}
%creating X^(n-k) = X^r
Xr = zeros(1,length(generatorPoly));   %All zero vector with length = r
Xr(end) = 1;          %Assign 1 to r'th position

%(X^r)m(X)
XrMX = gfconv(Xr, message);

%calculate the remainder(p(X)) of (X^r)m(X) / g(X) 
[qu p] = gfdeconv(XrMX, generatorPoly);
\end{lstlisting}

\item $c(X)$ is obtained by adding  $X^{n-k}m(X)$ and $p(X)$ (step 2 and 3 from previous section)

\begin{lstlisting}
%codeword calculated as c(X)= (X^r)m(X) + p(X)
codeword = mod([ p zeros(1, n - length(p))] + [XrMX zeros(1, n - length(XrMX))], 2);
\end{lstlisting}
\end{enumerate}



\section{Channel transmission}


\end{document} 